%-------------------------------------------------------------------------------
%	SECTION TITLE
%-------------------------------------------------------------------------------
\cvsection{Experiencia}


%-------------------------------------------------------------------------------
%	CONTENT
%-------------------------------------------------------------------------------
\begin{cventries}

%---------------------------------------------------------
  \cventry
    {Desarrollador web}
    {Enfok shop}
    {Guadalajara, México}
    {2014 - presente}
    {
      \begin{cvitems}
        \item {Puesta en marcha de un ecommerce para pagos en línea usando el stack: Saleor (Django), Nginx, Celery, Bootstrap, Jquery, dentro de un entorno Debian alojado en un Droplet de Digital Ocean. }
        \item {Añadir la opción de pagos en efectivo por medio de la generación de ordenes de pago en PDF usando weasyprint. }
        \item {Permitir pagos con Mercado Pago mediante la creación de una librería que integre la API de Mercado Pago y la librería django-payments. }
        \item {Adquisición de 1000 suscriptores en una semana, sin inversión en publicidad, al integrar un newsletter con sistema de referidos para email marketing. }
        \item {Mejorar las estadísticas, posterior a la sincronización de los eventos de la página web al pixel de Facebook.}
        %\item {Integrar sistemas de: calificación de productos, estimador de tiempos de entrega, historial de consultas, notificaciones, opiniones, manejo de códigos postales y colonias}
        %\item {Facilitar el mantenimiento de la Facebook store tras la automatización de la exportación de los productos en la base de datos a un archivo csv compatible}
        \item {Automatizar por completo la publicación de productos en grupos de facebook usando Selenium sin usar la API. }
        \item {Reducir el tiempo de creación de imágenes con precio y descripción individuales por medio de scripts usando la librería de Python Pillow, Image Magick y Bash.}
        \item {Mejorar las métricas de web performance en Lighthouse mediante la optimización de queries, orden de carga de los archivos y refactorización de código. Resultados 93, 98, 93 y 100 (Performance, Accessibility, Best Practices, SEO) en móvil y 98, 100, 93 y 100 en escritorio.}
      \end{cvitems}
    }

%---------------------------------------------------------
  \cventry
    {Blogger de desarrollo web}
    {Coffee Bytes}
    {Guadalajara, México}
    {Marzo 2019 - presente}
    {
      \begin{cvitems}
        \item {Escribo entradas principalmente sobre Django, GNU/Linux y Javascript.}
        \item {El despliegue del blog se hizo en un Droplet de Digital Ocean, de Ubuntu, usando WordPress como backend y el framework de React, Frontity, como frontend.}
      \end{cvitems}
    }

\end{cventries}

